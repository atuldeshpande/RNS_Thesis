% Thesis, Absatract
% by Rachel Slaybaugh

\noindent       % Don't indent this paragraph.
To enhance and improve the design of nuclear systems, high-fidelity neutron fluxes are required. Leadership-class machines provide platforms on which very large problems can be solved in a reasonable amount of time. Computing such fluxes accurately and efficiently requires numerical methods with good convergence properties and algorithms that can scale to hundreds of thousands of cores. Many 3-D deterministic transport codes are decomposable in space and angle only, limiting them to tens of thousands of cores. Most codes rely on methods such as Gauss Seidel for fixed source problems and power iteration wrapped around Gauss Seidel for eigenvalue problems, both of which can be slow to converge for challenging problems like those with highly scattering materials or high dominance ratios. 

\vspace*{0.5em}
\noindent       % Don't indent this paragraph.
Three methods have been added to the 3-D \Sn transport code Denovo that are designed to improve convergence and enable the full use of leadership-class computers. The first method is a multigroup Krylov solver that improves convergence when compared to Gauss Seidel and parallelizes the code in energy. Tests show that the multigroup Krylov solver can substantially outperform Gauss Seidel in challenging problems. The energy decomposition added by the solver allows Denovo to solve problems on hundreds of thousands of cores. 

\vspace*{0.5em}
\noindent       % Don't indent this paragraph.
The second method is Rayleigh quotient iteration (RQI), an old method being applied in a new context. This eigenvalue solver finds the dominant eigenvalue in a mathematically optimal way, and theory indicates that RQI should converge in fewer iterations than the traditional power iteration. RQI creates an energy-block-dense system that would be difficult for Gauss Seidel to solve. The new Krylov solver treats this kind of system very efficiently and RQI would not be a good choice without it. However, RQI creates poorly conditioned systems such that the method is only useful in very simple problems. Preconditioning can alleviate this concern. 

\vspace*{0.5em}
\noindent       % Don't indent this paragraph.
The final method is a multigrid in energy, physics-based preconditioner. Because the grids are in energy rather than space or angle, the preconditioner can easily and efficiently take advantage of the new energy decomposition. The new preconditioner was very effective at reducing multigroup iteration count for many types of problems. In some cases it also reduced eigenvalue iteration count. The application of the preconditioner allowed RQI to be successful for problems it could not solve otherwise. The preconditioner also scaled very well in energy, and was tested on up to 200,000 cores using a full-facility pressurized water reactor.

\vspace*{0.5em}
\noindent       % Don't indent this paragraph.
The three methods added to Denovo accomplish the goals of this work. They converge in fewer iterations than traditional methods and enable the use of hundreds of thousands of cores. Each method can be used individually, with the multigroup Krylov solver and multigrid-in-energy preconditioner being particularly successful on their own. For ``grand challenge'' eigenvalue problems, though, the largest benefit comes from using these methods in concert. 

